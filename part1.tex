\section{Normalize Representation}

Let a floating point system with base $\beta$, precision $p$, 
exponent range $[e_{\min}, e_{\max}]\cap\textbf{Z}$. Then a number in this 
floating point system can be represented in 
\textbf{normalize representation}: $a \times \beta^c$.
For example : 

\begin{enumerate}
    \item $0.1 = 1.00 \times 10^{-1}$ ($\beta = 10$, $p = 3$)
    \item $0.1 \approx 1.1001 \times 2^{-4}$ ($\beta = 2$, $p = 5$)
\end{enumerate}

Note that this respresentation cannot represent $0$,
since $a$ should always larger than $1$.
The natural representation for $0$ is $1.0 \times \beta^{e_{\min} - 1}$.

\section{Relative Errors and Unit of the Last Place(ulps)}

ulps, unit of the last place for short, use as a unit for representing error.
A system's ulps is $\beta^{-p+1}$.

Now we estimate the largest error on when we use a system to represent a number $r$:

\[ 
    E(r) = \min|a\times\beta^c - r|
\]

Assume $r = d_0.d_1d_2 \dots d_{p-1}d_p \dots \times \beta^c$,
we choose $s = d_0.d_1d_2 \dots d_{p-2}k \times \beta^c$,
where $k = d_{p-1}$ if $0.d_p \dots < \frac{\beta}{2}$ 
otherwise $k = d_{p-1} + 1$. Therefore $E(r)$ is bounded by $\frac{\beta}{2}\beta^{-p}\beta^c$.

Relative error is also be bounded:
\[
    \frac{E(r)}{r} \leq \frac{E(r)}{\beta^c} = \frac{\beta^{-p+1}}{2}
\]

This bound called machine epsilon $\epsilon = \frac{\beta^{-p+1}}{2}$.

\section{Guard Digits}

When we try to use computer $x - y$,
\textbf{compute and then round} is a method with less error. For example ($p = 3, \beta = 10$):

\begin{align}
    x     &= 2.1500000000 \nonumber \\
    y     &= 0.0000000125 \nonumber \\
    x - y &= 2.1499999875 \nonumber
\end{align}

Round to $2.15$.

On the other hand, \textbf{round and then compute}
method can get same answer in more elegent way.

\begin{align}
    x     &= 2.15 \nonumber \\
    y     &= 0.00 \nonumber \\
    x - y &= 2.15 \nonumber
\end{align}

However, in some cases, \textbf{round and then compute}
can cause large relative error.

\begin{align}
    x      &= 1.00  \nonumber \\
    y      &= 0.999 \nonumber \\
    y'     &= 0.99  \nonumber \\
    x - y' &= 0.01  \nonumber \\
    x - y  &= 0.001 \nonumber
\end{align}

Relaive Error = $\frac{|0.01 - 0.001|}{0.001} = 9$.

\begin{theorem}
    Using $p$ digits with base $\beta$ for $x-y$, the relative error
    can be as large as $\beta - 1$.
\end{theorem}

Large errors occur if x and y are close, and \textbf{single guard digit}
can porevent this situation:
{\it $p$ increased by 1 in the device for addition and subtraction.}


\begin{theorem}
    Using $p+1$ digits for $x-y$ $\Rightarrow$ relative error $ < 2\epsilon$
\end{theorem}

\begin{proof}
    Let $y'$ be truncated $y$, $\delta = E(x - y')$.

    \emph{Case 1: $x - y' \geq 1$}

        $x - y = (x - y') - (y - y') \geq 1 + \beta^p - \beta^p = 1$

    \emph{Case 2: $x - y' < 1$}

        $\delta = 0$

\end{proof}

\section{Cancellation}

Consider $a = 1.22, b = 3.34, c = 2.28$, then $b^2 - 4ac = 0.0292$.

But

\begin{align}
    b^2       &\approx 11.1 \nonumber \\
    4ac       &\approx 11.2 \nonumber \\
    b^2 - 4ac &\approx  0.1 \nonumber
\end{align}

When we calculate $b^2 - 4ac$, $b^2$ and $4ac$ are already contain error,
which causes a large error. This situation called \textbf{Catastrophic cancellation}.

How to avoid \textbf{Catastrophic cancellation} ?

\subsection{$\frac{-b + \sqrt{b^2 - 4ac}}{4ac}$}

When $b > 0, b^2 \gg 4ac$, then $\sqrt{b^2 - 4ac} - b \approx 0$.
By reformulate it, we can get $\frac{2c}{- b - \sqrt{b^2 - 4ac}}$,
which has less Catastrophic cancellation.

Note that we might not be able to remove all Catastrophic cancellation.

\subsection{$x^2 - y^2$}

When $x$ and $y$ are close, $(x-y)(x+y)$ is better than $x^2 - y^2$.

\section{Rounding}

Computing the nearest value are usually accurate.
For instance, $1.4$ round to $1$ and  $1.6$ round to $2$.

\begin{theorem}
    Let $x_0$ = $x$, $x_1 = (x_0-y)+y$, $\dots$ , $x_n = (x_{n-1}-y)+y$.
    If $+$ and $-$ operation use rounding even, then
    $x_i = x_0 \ \forall \ i \in [1, n]$ or 
    $x_i = x_1 \ \forall \ i \in [1, n]$
\end{theorem}

\section{IEEE Standard}

How to choose $\beta$ and $p$ ?

\begin{enumerate}
    \item $\beta = 16, p = 1 \Rightarrow \epsilon = \frac{16}{2}16^{-1} = \frac{1}{2}$ 
    \item $\beta = 2, p = 4 \Rightarrow \epsilon = \frac{2}{2}2^{-4} = \frac{1}{16}$ 
\end{enumerate}

\section{Special Quantities}

\subsection{Infinity}

Consider \textbf{Infinity Arthmetic}: $+0$, $-0$, $+\infty$, $-\infty$.
These thing is to represent something is too large or too small such that
the floating point representation in the machine cannot respresent.
Calculation rule is following the limit calculation, for example:
$\lim_{x \rightarrow +0} \frac{1}{x} = +\infty$.

\subsection{Not a Number}
For something calculation we don't have a real number for it, we would use {\it Nan}.
For example :

\begin{align}
    \sqrt{-4} &= \it{Nan}       \nonumber \\
    \infty - \infty &= \it{Nan} \nonumber
\end{align}

\subsection{$+0$ and $-0$}

IEEE defines $+0 = -0$. This definition do have some pros.
However it still causes some problem:

\[
    \log x = 
    \left \{
        \begin{matrix}
            -\infty   & x = +0  \\
            \it{Nan} & x = -0  \\
        \end{matrix}
    \right.
\]

and $+0 = -0$ while $\frac{1}{+0} \neq \frac{1}{-0}$.

\subsection{Denormalized number}

For smaller number : $0.d_1d_2 \dots \times 2^{e_{\min}}$.

\section{Handle the exception or trap}

Sometimes, we want to handle the number when special quantities appear rather than let it go.
The compiler should give programmer this option :
\textbf{Use Special Number} or \textbf{Trap}.

\begin{center}
    \begin{tabular}{ | l | l |}
        \hline
        Stituation & No exception  \\ \hline
        overflow   & $\pm \infty$  \\ \hline
        underflow  & 0             \\ \hline
        $/0$       & $\infty$      \\ \hline
        invalid    & {\it Nan}     \\ \hline 
        inexact    & round(x)      \\ \hline
    \end{tabular}
\end{center}

\subsection{Example : Power Function}
When $n < 0$, calculate pow(x, -n) or pow(1/x, n).
Choose the result which didn't trigger overflow or underflow execption.

\subsection{Example : Calculate arccos}


