\documentclass{article}
\title{Numerical Methods Notes}
\author{Mudream}
\date{\today}
\begin{document}
    \maketitle
    \part{Floating Point Operation}
        \section{Normalize representation}

        Let a floating point system with base $\beta$, precision $p$, 
        exponent range $[e_{\min}, e_{\max}]\cap\textbf{Z}$. Then a number in this 
        floating point system can be represented in 
        \textbf{normalize representation}: $a \times \beta^c$.
        For example : 
        
        \begin{enumerate}
            \item $0.1 = 1.00 \times 10^{-1}$ ($\beta = 10$, $p = 3$)
            \item $0.1 \approx 1.1001 \times 2^{-4}$ ($\beta = 2$, $p = 5$)
        \end{enumerate}

        Note that this respresentation cannot represent $0$,
        since $a$ should always larger than $1$.
        The natural representation for $0$ is $1.0 \times \beta^{\min - 1}$
        
\end{document}

